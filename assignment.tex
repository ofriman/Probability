\documentclass[a4paper, 11pt, oneside]{article}

 \usepackage[margin=1in]{geometry} 
\usepackage{amsmath,amsthm,amssymb, float,centernot}
\usepackage[shortlabels]{enumitem}
 \usepackage[hidelinks]{hyperref}
\usepackage{xcolor}
\hypersetup{
    colorlinks,
    linkcolor={red!50!black},
    citecolor={blue!50!black},
    urlcolor={blue!80!black}
}

\newtheorem{theorem}{Theorem}[section]
  
\newcommand{\N}{\mathbb{N}}
\newcommand{\Z}{\mathbb{Z}}
\newcommand{\R}{\mathbb{R}}
\newcommand\abs[1]{\left|#1\right|}

\DeclareRobustCommand{\bbone}{\text{\usefont{U}{bbold}{m}{n}1}}

\DeclareMathOperator{\EX}{\mathbb{E}}
\DeclareMathOperator{\PX}{\mathbb{P}}
\begin{document}

\title{Final take-home exam}
\author{Oren Friman 301677613}
\maketitle
				   
\section{Introduction}
\begin{enumerate}
\item
 \begin{enumerate}
\item  In the solution of this question we will use the following version of Chernoff inequality which was proved in the lecture.
\begin{theorem} (Chernoff inequalities). Let $X_1,\ldots,X_n$ be independent random variabbles that return values in $[0,1]$, and let $\sum_{i=1}^nX_i$. Then
 \begin{enumerate}
 \item For every $t>0$ it holds that
 \begin{equation*}
P(X\geq\EX(X)+t) \leq e^{\frac{-2t^2}{n}}\text{ and } P(X\leq\EX(X)-t) \leq e^{\frac{-2t^2}{n}}
\end{equation*}
 \end{enumerate}
\end{theorem}
For every $i$, let $Y_i = \frac{X_i+1}{2}$ and $Y = \sum_{i=1}^nY_i$. Then $Pr(Y_i=0) = Pr(Y_i=\frac{1}{2}) = Pr(Y_i=1) = \frac{1}{3}$ for every $i$. We have
  \begin{equation}\label{ex}
 \EX(X_i) = -1 \frac{1}{3} + 0 \frac{1}{3} + 1 \frac{1}{3} = 0,
 \end{equation}
\begin{equation*}
 \EX(Y_i) = \EX\bigg(\frac{X_i + 1}{2}\bigg) = \frac{1}{2}\EX(X_i)+\frac{1}{2} = \frac{1}{2},
 \end{equation*}
  \begin{equation*}
 \EX(Y) = \EX\bigg(\sum^n_{i=1}Y_i\bigg) = \frac{n}{2},
 \end{equation*}
due to the linearity of the expectation. Therefore
\begin{align*}
Pr(\abs{S_n}>2\sqrt{n}) = Pr(S_n<-2\sqrt{n}) + Pr(S_n>2\sqrt{n}),
\end{align*}
Let's calculate separately
\begin{align*}
 Pr(S_n<-2\sqrt{n}) &=
 Pr\bigg(\sum_{i=1}^nX_i<-2\sqrt{n}\bigg) \\&=
 Pr\bigg(\sum_{i=1}^n\frac{X_i+1}{2}<\frac{-2\sqrt{n}+n}{2}\bigg)\\&=
 Pr\bigg(\sum_{i=1}^nY_i<\frac{-2\sqrt{n}+n}{2}\bigg)\\&=
 Pr\bigg(Y<\frac{n}{2} - \sqrt{n}\bigg)\\&=
  Pr\bigg(Y<\EX(Y) - \sqrt{n}\bigg) \leq e^{\frac{-2(\sqrt{n})^2}{n}} = e^{-2},
\end{align*}
Similarly
\begin{align*}
 Pr(S_n>2\sqrt{n}) &=
 Pr\bigg(\sum_{i=1}^nX_i>2\sqrt{n}\bigg) \\&=
 Pr\bigg(\sum_{i=1}^n\frac{X_i+1}{2}>\frac{2\sqrt{n}+n}{2}\bigg)\\&=
 Pr\bigg(\sum_{i=1}^nY_i>\frac{2\sqrt{n}+n}{2}\bigg)\\&=
 Pr\bigg(Y>\frac{n}{2} + \sqrt{n}\bigg)\\&=
  Pr\bigg(Y>\EX(Y) + \sqrt{n}\bigg) \leq e^{\frac{-2(\sqrt{n})^2}{n}} = e^{-2},
\end{align*}
So
\begin{equation*}
Pr(\abs{S_n}>2\sqrt{n}) \leq 2 e^{-2}.
 \end{equation*}
 \item  In the solution of this question we will use the central limit theorem which states the following.
\begin{theorem} 
\label{central}
(the Central Limit Theorem). Let $\{X_i\}^\infty_{i=1}$ be a sequence of mutually independent identically distributed random variables 
with finite expectation $\mu$ and finite variance $\sigma^2>0$. For every positive integer $n$, let $Y_n = \frac{X_1 + \ldots + X_n - n \mu}{\sigma\sqrt{n}}$
and let $F_n$ be the cumulative distribution function of $Y_n$, that is, $F_n(\alpha) = \PX(Y_n\leq\alpha)$ for every $\alpha\in \R$. 
Then $\lim_{n \to \infty} F_n(\alpha) = \Phi(\alpha)$ holds for every $\alpha \in \R$
\end{theorem} 
Let's calculate the expectation and variance, recall from \ref{ex} that $\EX(X_i) = 0$, we have
\begin{equation*}
\EX(X_i^2) = (-1)^2 \frac{1}{3} + (0)^2 \frac{1}{3} + (-1)^2 \frac{1}{3} = \frac{2}{3},
 \end{equation*}
\begin{equation*}
Var(X_i) = \EX(X_i^2) + (\EX(X_i))^2 = \frac{2}{3}.
 \end{equation*}
 Let $Y = \frac{S_n}{\sqrt{\frac{2}{3}}}$,
since the $X_i$`s are independent and identically distributed random vairables with finite expectation and variance, 
 we can apply Therom \ref{central} to obtain
 \begin{align*}
Pr(\abs{S_n}>2\sqrt{n}) &=
 1 - Pr(\abs{S_n} \leq 2\sqrt{n}) \\&=
 1 - Pr(-2\sqrt{n} \leq S_n \leq 2\sqrt{n}) \\&=
 1 - Pr\Bigg(\frac{-2\sqrt{n} - n \cdot 0}{\sqrt{\frac{2}{3}}\sqrt{n}} \leq \frac{S_n - n \cdot 0}{\sqrt{\frac{2}{3}}\sqrt{n}} \leq \frac{2\sqrt{n} - n \cdot 0}{\sqrt{\frac{2}{3}}\sqrt{n}}\Bigg) \\&=
 1 - Pr\Bigg(\frac{-2}{\sqrt{\frac{2}{3}}} \leq \frac{S_n}{\sqrt{\frac{2}{3}}} \leq \frac{2}{\sqrt{\frac{2}{3}}}\Bigg)\\&=
 1 - Pr\Bigg(-\sqrt{6} \leq Y \leq \sqrt{6}\Bigg)\\&=
 1 - \Big[Pr(Y \leq \sqrt{6}) - Pr(Y < -\sqrt{6})\Big] \\&=
 1 - \Big[\Phi(\sqrt{6}) -\Phi(-\sqrt{6})\Big] \\&=
 1 - \Big[\Phi(\sqrt{6}) - (1 - \Phi(\sqrt{6}))\Big] \\&=
 2 - 2\Phi(\sqrt{6})
 \end{align*}
\end{enumerate}

\end{enumerate}

\begin{thebibliography}{9} 

\end{thebibliography}

\end{document}